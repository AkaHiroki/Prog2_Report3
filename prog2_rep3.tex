\documentclass[a4paper, 11pt, titlepage]{jsarticle}
\usepackage[dvipdfmx]{graphicx}
\usepackage{amsmath}
\usepackage{listings}
\usepackage{comment}
\usepackage{ascmac}

\title{プログラミング2\\ 課題レポート3: リファクタリングを通してユニットテストとバージョン管理に慣れよう}
\author{225743C 赤嶺 紘基}
\date{\today}

\begin{document}
\maketitle

\section{課題概要}
\begin{itembox}[l]{課題概要}
・コード例ex-gradle2023は、ターン制バトルゲーム(1ターン毎に勇者とスライムが戦い、どちらかが倒れるまで攻撃し合う)を書いている途中だ。ただし次に示す問題があるため、修正したい。ユニットテストとバージョン管理システムを使いながら、一歩ずつリファクタリングしてみよう。\\

\textbf{状況}:各ターンではまず勇者が敵を攻撃し、ダメージに応じてスライムのHPを修正する。次にスライムが勇者を攻撃し、勇者のHPを修正する。これを繰り返すが、どちらかが倒れたらループを抜ける。\\
\textbf{問題}:勇者がスライムを倒しても、スライムが勇者を攻撃できてしまう。\\

・問題を確認するテストコード(EnemyTest.java)。
次の手順で問題の有無を確認している。

(1) ヒーローと敵を準備。ヒーローの攻撃力は敵を一撃で倒せるほど強い状態とする。具体的にはヒーローの攻撃力を100とし、スライムのHPを10とする。\\

(2) ヒーローが殴り、敵も殴る。敵は一撃で倒されていることを期待。実際には乱数を用いているためたまたま倒れないこともあるが、殆どの場合で即死ダメージを与える。\\

(3) 敵が死んだ状態ならば攻撃できないはず。つまり攻撃実行してもヒーローのHPは減っていないことを期待する。この期待通りになるかどうかをコードで検証する。\\

初期HPを保存している変数(defaultHeroHp)と、殴られた後のデモ勇者のHP(demoHero.hitPoint)とが等しいならば、減っていないと見做せる。\\
\end{itembox}

\section{STEP1について}
pushしたリポジトリURL(Git URL)\\


\section{STEP2について}
指定手順で得られる EnemyTest.attack()の差分コードを掲載し、どのように修正したのか解説せよ。\\


\section{STEP3について}
今回のコード修正を通して、気づいたことを報告せよ。\\


\section{STEP4について}
キャプチャ画像を掲載せよ。\\














\end{document}